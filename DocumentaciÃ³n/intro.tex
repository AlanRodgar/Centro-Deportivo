%---------------------------------------------------------
\begin{description}
	\item
En este documento está contenido el análisis para el desarrollo del sistema encargado de controlar el acceso que tienen los afiliados a las instalaciones, automatizar el ingreso a las áreas por parte de éstos. Además de tener un registro de los servicios, áreas, cursos y de los trabajadores con los que cuenta cada sucursal de la cadena de gimnasios {\bf Centro Deportivo San Pancho }. El sistema tendrá los módulos que podrán registrar pagos de usuarios que compren membresías.
	
Este documento ilustra las técnicas que se utilizan para desarrollar un proyecto de software en sus diferentes etapas, y permitir una comprensión más profunda de los conceptos del desarrollo del sistema. La mayor parte del contenido que incluye este documento se presenta en forma escrita, de manera que podamos ver de manera clara en que consiste cada módulo de desarrollo, los cuales están definidos por:

	\item[Project Charter:] Es el documento donde se muestra el propósito del proyecto y su descripción, las necesidades que requiere el usuario, para que el sistema sea funcional y permita resolver el problema, así como el alcance, tiempo de desarrollo y costos.
	\item[Modelo del comportamiento:] Se define cada una de las interacciones que tiene el sistema con los usuarios, mostrando los objetivos de desarrollar dicha interacción, los pasos para realizar las operaciones y sus respectivas restricciones y mensajes de error.
	\item[Modelado de actores:] Defina la jerarquía de los usuarios del sistema y los privilegios que tiene cada uno dentro del mismo
	\item[Modelo del dominio del problema:] Especifica las relaciones que existen dentro del sistema mediante una base de datos, para que los registros sean almacenados. Así como un diccionario de datos donde se especifica cada uno de los atributos, estos se expresan en un lenguaje sencillo para poder entender su significado.
	\item[Modelo de interacción con el usuario:] Muestra la navegación por las diferentes secciones con las que cuenta el sistema, de acuerdo a los diferentes tipos de usuarios.
\end{description}

%---------------------------------------------------------
\newpage
\section{Definición del problema}

\begin{description}
	\item[Control de acceso] El Centro Deportivo San Pancho utiliza libretas para mantener los registros de los usuarios que acceden a las instalaciones de las diferentes sucursales. Este técnica para llevar el control del acceso a las áreas puede ser vulnerable debido a la perdida o los daños que le puedan suceder a la libreta, afectando la información de la empresa. Además se sabe que existen afiliados que no han renvado su membresía y siguen accediendo a los servicios y áreas de las sucursales.
	
	\item[Almacenamiento de la información]  El almacenamiento de la información de un cliente referente a su membresía, esto es, que esté vigente, la fecha en que la adquirió, el tipo de membresía con la que cuenta y si está pendiente el pago de su mensualidad. Estos registros se realizan en papel y se archivan el carpetas. En ocaciones se han desprendido hojas de la carpeta ocacionando que se pierda el registro de uno o más clientes y por ende la molestia de éstos.
	
	\item[] Respecto a los cursos. 
\end{description}












