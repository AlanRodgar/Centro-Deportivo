%---------------------------------------------------------
\section{Glosario de Términos del Negocio}

\begin{description}
	\item[Login:] Sección del sistema que auténtica al usuario mediante un username y password, permitiéndonos identificar su tipo (cliente, usuario, recepcionista, ejecutivo de ventas, gerente de sucursal, gerente de operaciones, encargado de área e instructor) brindándole acceso a su perfil.
	\item[Username:] Nombre único que identifica a cada usuario dentro del sistema.
	\item[Password:] Clave de acceso conformada por carácteres alfanuméricos asociada a un username.
	\item[Usuario del sistema:] Persona que interactúa con el sistema.  
	\item[Cliente:] Persona registrada en el sistema que genera el pago de la memebresia. Este puede actuar también como usuario.
	\item[Membresia:] Contrato que específica los beneficios a los que el usuario tiene acceso generando un costo a partir de los mismos. Esos pueden ser cursos y servicios.
	\item[Servicio:] Recursos materiales y de infraestructura a los que el usuario tiene derecho por el pago de su membresía.
	\item[Curso:] Actividad deportiva que el usuario práctica por el pago de su membresía.
	\item[Area:] Lugar dentro de la sucursal donde se imparten cursos.
	\item[Sucursal:] Establecimiento del centro deportivo.
	\item[Horario:] Distribición de las horas en que se imparte un curso.
	\item[Cupo: ] Capacidad mpaxima dentro de un área o curso.
	\item[Cliente Moral:] Persona que se registra en el sistema como empresa.
	\item[Usuario:] Persona que utiliza las áreas, servicios y cursos del centro deportivo.
	\item[Recepcionista:] Persona que se ubica en la entrada de cada sucursal y da la bienvenida al cliente y/o Usuario.
	\item[Gerente de Sucursal:] Persona encargada de cada sucursal.
	\item[Gerente de operaciones:] Persona encargada del centro deportivo como conjunto.
	\item[Encargado de area:] Persona encargada de las áreas de sucursal.
	\item[Ejecutivo de ventas:] Persona que vende membresías.
	\item[Instructor:] Persona que imparte cursos.
	\item[CVV2:] Ultimos tres digitos de la parte trasera de la tarjeta de crédito y/o débito.
	\item[Básica:] Tipo de membresía.
	\item[Plus:] Tipo de membresía.
	\item[Premium:] Tipo de membresía.
\end{description}

%---------------------------------------------------------
\section{Hechos del Negocio}

\begin{description}
	\item[Pago {\em de} un Cliente:]
\end{description}

%---------------------------------------------------------
\section{Reglas de Negocio}

\begin{BussinesRule}{BR1}{Determinar si un cliente puede comprar una membresia.} 
	\BRitem[Descripción:] El cliente deberá ser mayor de edad para adquirir una membresia.
	\BRitem[Tipo:] Habilitadora.
	\BRitem[Nivel:] Estricta.
\end{BussinesRule}

\begin{BussinesRule}{BR2}{Determinar si un usuario puede inscribir un curso.}
	\BRitem[Descripción:] El usuario deberá tener o estar afiliado a una membresia y existir disponiblidad de cupo.
Esto podría generar un costo adicional al pago de su membresia.
	\BRitem[Tipo:] Habilitadora.
	\BRitem[Nivel:] Estricta.
\end{BussinesRule}

\begin{BussinesRule}{BR3}{Validar el horario del curso inscrito.}
	\BRitem[Descripción:] El horario del curso inscrito por el usuario no puede tener traslapes con otros cursos.
	\BRitem[Tipo:] Habilitadora.
	\BRitem[Nivel:] Estricta.
\end{BussinesRule}

\begin{BussinesRule}{BR4}{Tiempo de tolerancio en el acceso a un curso.}
	\BRitem[Descripción:] El usuario no podrá acceder al curso con mas 20 minutos iniciado el curso. 
	\BRitem[Tipo:] Ejecutiva.
	\BRitem[Nivel:] PostAutorizada.
\end{BussinesRule}

\begin{BussinesRule}{BR5}{Upgrade de membresía.}
	\BRitem[Descripción:] El usuario podrá subir de categoría siempre y cuando no tenga una membresía de tipo premium.
	\BRitem[Tipo:] Ejecutiva.
	\BRitem[Nivel:] PostAutorizada.
\end{BussinesRule}

\begin{BussinesRule}{BR6}{Horario de servicio del gimnasio.}
	\BRitem[Descripción:] Los usuarios solo podrán ingresar de 6:00 am a las 10:00 pm.
	\BRitem[Tipo:] Habilitadora.
	\BRitem[Nivel:] Estricta.
\end{BussinesRule}

\begin{BussinesRule}{BR7}{Validar horario de instructor.}
	\BRitem[Descripción:] Un instructor no puede estar ascociado a dos cursos en el mismo horario.
	\BRitem[Tipo:] Habilitadora.
	\BRitem[Nivel:] Estricta.
\end{BussinesRule}

\begin{BussinesRule}{BR8}{Determinar la contratación de un empleado}
	\BRitem[Descripción:] No se puede contratar a un empleado que sea menor de edad.
	\BRitem[Tipo:] Habilitadora.
	\BRitem[Nivel:] Estricta.
\end{BussinesRule}

\begin{BussinesRule}{BR9}{Pago de membresía}
	\BRitem[Descripción:] El pago de la membresia deberá ser en una sola exhibicion ya sea en efectivo o con tarjeta de debito y/o tarjeta de debito.
	\BRitem[Tipo:] Habilitadora.
	\BRitem[Nivel:] Estricta.
\end{BussinesRule}

\begin{BussinesRule}{BR10}{Registro de áreas}
	\BRitem[Descripción:] Una área solo puede registrarse a menos que haya espacio suficiente en la sucursal.
	\BRitem[Tipo:] Habilitadora.
	\BRitem[Nivel:] Estricta.
\end{BussinesRule}

\begin{BussinesRule}{BR11}{Registro de áreas}
	\BRitem[Descripción:] Una área solo puede registrarse a menos que haya espacio suficiente en la sucursal.
	\BRitem[Tipo:] Habilitadora.
	\BRitem[Nivel:] Estricta.
\end{BussinesRule}

\begin{BussinesRule}{BR12}{Costo de membresía}
	\BRitem[Descripción:] El costo de las membresias está determinado por el numero de cursos y/o servicios que el usuario adquiera, distribiudos de la siguiente manera:

- Basica: 
		Servicios:Locker, regaderas.			
			Areas: Gimnasio
			cursos: Ninguno

- Plus: 
		Servicios:Locker, regaderas, alberca, spa, nutición.
			Areas: Gimnasio
			cursos: 1 Curso

- Premium: 
		Servicios:Locker, regaderas, alberca, spa, nutición, área de bronceado.
			Areas: Gimnasio
			cursos: 2 Cursos
	\BRitem[Tipo:] Habilitadora.
	\BRitem[Nivel:] Estricta.
\end{BussinesRule}

\begin{BussinesRule}{BR13}{Costo de curso.} 
	\BRitem[Descripción:] El costo de los cursos estarán definidos por su duración, dias de la semana en que se imparten, horario en que se imparten y el numero de instructores capacitados para impartir dicho curso.
	\BRitem[Tipo:] Ejecutiva.
	\BRitem[Nivel:] PreAutorizada.
\end{BussinesRule}

\begin{BussinesRule}{BR14}{Costo de servicios adicionales a la membresía.} 
	\BRitem[Descripción:] El costo de los servicios estarán definidos por la capacidad de recursos y materiales e infraestructura en el momento de la adquisisión.
	\BRitem[Tipo:] Ejecutiva.
	\BRitem[Nivel:] PreAutorizada.
\end{BussinesRule}

\begin{BussinesRule}{BR15}{Costo de servicios adicionales a la membresía.} 
	\BRitem[Descripción:] El costo de los cursos estarán definidos por su duración, dias de la semana en que se imparten, horario en que se imparten y el numero de instructores capacitados para impartir dicho curso.
	\BRitem[Tipo:] Ejecutiva.
	\BRitem[Nivel:] PreAutorizada.
\end{BussinesRule}

\begin{BussinesRule}{BR16}{Elección de un instructor para impartir un curso.} 
	\BRitem[Descripción:] Para que un instructor sea asociado a un curso este deberá poseer una certificación.
	\BRitem[Tipo:] Ejecutiva.
	\BRitem[Nivel:] PreAutorizada.
\end{BussinesRule}

\begin{BussinesRule}{BR17}{Registro de cursos.} 
	\BRitem[Descripción:] Para el registro de nuevos cursos se debe contar con el área disponible en la sucursal que cuente con las caracteristicas ideales para el mismo.
	\BRitem[Tipo:] Cronometrada.
	\BRitem[Nivel:] PreAutorizada.
\end{BussinesRule}

\begin{BussinesRule}{BR117}{Determinar si el usuario tiene acceso al sistema.} 
	\BRitem[Descripción:] Un Gerente de operación de negocio necesita estar dado de altra en el sistema e introducir su usuario y contraseña de manera correcta.
	\BRitem[Tipo:] Habilitadora.
	\BRitem[Nivel:] Estricta.
\end{BussinesRule}

\begin{BussinesRule}{BR118}{Determinar si los datos de los campos de un formulario son del tipo adecuado} 
	\BRitem[Descripción:] Los campos pueden ser del tipo cadena de carácteres, valores númericos y valores alfanúmericos
	\BRitem[Tipo:] Habilitadora.
	\BRitem[Nivel:] Estricta.
\end{BussinesRule}


