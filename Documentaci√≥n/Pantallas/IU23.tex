\subsection{IU18 Pantalla de menú de ocpiones de la sección áreas.}

\subsubsection{Objetivo}
	El gerente podrá seleccionar una opcion del menú areas.

\subsubsection{Diseño}
	Esta pantalla se encuentra en la seccion de opciones del menu áreas de la sección principal. 

\IUfig[.5]{gui/menuOpciones}{IU18}{Pantalla de Menú de opciones.}

\subsubsection{Salidas}

	Ninguna.

\subsubsection{Entradas}
El gerente podrá escoger alguna opción del menú con tan solo dar un click.

\subsubsection{Comandos}
\begin{itemize}
	\item \IUbutton{Registrar áreas}: Muestra un formulario para registrar áreas en la sucursal.
	\item \IUbutton{Consultar áreas}: Muestra las áreas registradas en la sucursal.
	\item \IUbutton{Eliminar áreas}: Muestra una lista desplegable donde el gerente puede eliminar las áreas registradas.
	\item \IUbutton{Actualizar áreas}: Muestra un formulario en donde se pueden modificar los datos de las áreas.
\end{itemize}

\subsubsection{Mensajes}
	\begin{Citemize}
		\item {\bf MSG0} Ningun mensaje de error.
	\end{Citemize}

%%%%%%%%%%%%%%%%%%%%%%%%%%%%%%%%%%%%%%%%%%%%%%%%%%%%%%%%%%%%%%%%%%%%%%%%%%%%%%%%%	
%%%%%%%%%%%%%%%%%%%%%%%%%%%%%%%%%%%%%%%%%%%%%%%%%%%%%%%%%%%%%%%%%%%%%%%%%%%%%%%%%
\subsection{IU19 Pantalla de registro de área}

\subsubsection{Objetivo}
	El gerente podrá registrar nuevas áreas en la sucursal.

\subsubsection{Diseño}
	Esta pantalla en la opción registrar áreas que se encuentra en la seccion de opciones del menu áreas de la sección principal.

\IUfig[.5]{gui/registrarArea}{IU19}{Pantalla de registo de nueva área.}

\subsubsection{Salidas}
Mensaje de área registrada exitosamente.

\subsubsection{Entradas}
Nombre de área, Tipo de área, Largo, Ancho, Capacidad, Responsable y Descripción.

\subsubsection{Comandos}
\begin{itemize}
	\item \IUbutton{Registrar área}: Es la acción para dar de alta una nueva área.
	\item \IUbutton{Limpiar}: Limpia el formulario, deja los campos en blanco.
\end{itemize}

\subsubsection{Mensajes}
	\begin{Citemize}
		\item {\bf MSG1} Campo nombre: Debe contener letras y espacios en blanco.
	\end{Citemize}
%%%%%%%%%%%%%%%%%%%%%%%%%%%%%%%%%%%%%%%%%%%%%%%%%%%%%%%%%%%%%%%%%%%%%%%%%%%%%%%%%	
%%%%%%%%%%%%%%%%%%%%%%%%%%%%%%%%%%%%%%%%%%%%%%%%%%%%%%%%%%%%%%%%%%%%%%%%%%%%%%%%%
\subsection{IU20 Pantalla de Eliminar áreas}

\subsubsection{Objetivo}
	El gerente podrá dar de baja alguna área que se encuentre registrada.

\subsubsection{Diseño}
	Esta pantalla se encuentra en la sección de áreas del menú principal.

\IUfig[.5]{gui/eliminarAreas}{IU20}{Pantalla de eliminar áreas.}

\subsubsection{Salidas}

	Mensaje de registro eliminado correctamete.

\subsubsection{Entradas}
Entrada directamente sobre el mouse, en la opción eliminar de la lista desplegable de las áreas.

\subsubsection{Comandos}
\begin{itemize}
	\item \IUbutton{Eliminar}: Esta opción permite eliminar un área registrada.
	\item \IUbutton{Regresar al menú principal}: Esta opción permite regresar al menú principal.
\end{itemize}

\subsubsection{Mensajes}
	\begin{Citemize}
		\item {\bf MSG5} Ningun mensaje de error.
	\end{Citemize}
%%%%%%%%%%%%%%%%%%%%%%%%%%%%%%%%%%%%%%%%%%%%%%%%%%%%%%%%%%%%%%%%%%%%%%%%%%%%%%%%%	
%%%%%%%%%%%%%%%%%%%%%%%%%%%%%%%%%%%%%%%%%%%%%%%%%%%%%%%%%%%%%%%%%%%%%%%%%%%%%%%%%

\subsection{IU22 Pantalla de Actualización de áreas}

\subsubsection{Objetivo}
	El gerente podrá actualizar una área qeu se encuentre registrada a fin de modificar su contenido.

\subsubsection{Diseño}
	Esta pantalla se encuentra en la sección de áreas del menú principal.

\IUfig[.5]{gui/actualizaArea}{IU22}{Pantalla de eliminar áreas.}

\subsubsection{Salidas}

	Mensaje de datos actualizados correctamente.

\subsubsection{Entradas}
En caso de ser necesario el gerente podrá actualizar los datos: Nombre de área, Tipo de área, Largo, Ancho, Capacidad, Responsable y Descripción.

\subsubsection{Comandos}
\begin{itemize}
	\item \IUbutton{Registrar área}: Es la acción para dar de alta una nueva área.
	\item \IUbutton{Limpiar}: Limpia el formulario, deja los campos en blanco.
\end{itemize}

\subsubsection{Mensajes}
	\begin{Citemize}
		\item {\bf MSG5} Ningun mensaje de error.
	\end{Citemize}

%%%%%%%%%%%%%%%%%%%%%%%%%%%%%%%%%%%%%%%%%%%%%%%%%%%%%%%%%%%%%%%%%%%%%%%%%%%%%%%%%	
%%%%%%%%%%%%%%%%%%%%%%%%%%%%%%%%%%%%%%%%%%%%%%%%%%%%%%%%%%%%%%%%%%%%%%%%%%%%%%%%%

\subsection{IU23.0 Pantalla de Control de acceso Gerente de Sucursal}

\subsubsection{Objetivo}
	Controlar el acceso al sistema mediante una contraseña a fin de que cada usuario acceda solo a las operaciones permitidas para su perfil.

\subsubsection{Diseño}
	Esta pantalla aparece al iniciar el sistema. Para ingresar al mismo se debe escribir el nombre de usuario y contraseña del gerente de sucursal. 

\IUfig[.5]{gui/iu21-0}{IU21.0}{Pantalla de Control de Acceso a Gerente de Sucursal.}

\subsubsection{Salidas}

	Ninguna.

\subsubsection{Entradas}
Número de Gerente de sucursal y contraseña unica.

\subsubsection{Comandos}
\begin{itemize}
	\item \IUbutton{Entrar}: Verifica que el Gerente de Sucursal se encuentre registrado y la contraseña sea la correcta. Si la verificación es correcta, se muestra la \IUref{IU21.1}{Pantalla de inicio acceso a Gerente de sucrsal}.
	\item \IUbutton{Ayuda}: Muestra la ayuda de esta pantalla \IUref{IU50}{Pantalla de Ayuda}.
\end{itemize}

\subsubsection{Mensajes}
	\begin{Citemize}
		\item {\bf MSG5} Error al verificar los datos de Gerente de sucursal, vuelva a intentarlo.
	\end{Citemize}

