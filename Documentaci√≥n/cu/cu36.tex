% \IUref{IUAdmPS}{Administrar Planta de Selección}
% \IUref{IUModPS}{Modificar Planta de Selección}
% \IUref{IUEliPS}{Eliminar Planta de Selección}


% Copie este bloque por cada caso de uso:
%-------------------------------------- COMIENZA descripción del caso de uso.


%\begin{UseCase}[archivo de imágen]{UCX}{Nombre del Caso de uso}{
	\begin{UseCase}{CU36}{Actualizar Servicios.}{
		Permite al gerente de sucursal, recepcionista, actualizar los datos que crea necesarios con el fin de corregir errores en los datos registrados o posibles modificaciones.
	}
		\UCitem{Versión}{1.0}
		\UCitem{Actor}{Gerente de Sucursal, Recepcionista}
		\UCitem{Propósito}{Hacer modificaciónes en los servicios registrados a fin de corregir datos o alguna actualización en ellos.}
		\UCitem{Entradas}{Nombre, ubicación, estado del servicio, costo, horario, responsable y descripción.}
		\UCitem{Origen}{Desde el teclado.}
		\UCitem{Salidas}{Se muestra un mensaje de "servicio actualizado".}
		\UCitem{Destino}{Los datos serán almacenados en el sistema y se verán reflejados en el área de consulta de servicios.}
		\UCitem{Precondiciones}{
		Que exista el servicio registrado previamente para hacer modificaciones.
		Se requiere que el servicio no se haya eliminado del sistema.
		Se requiere que no se haya eliminado totalmente todas las áreas e instructores.}

		\UCitem{Postcondiciones}{El actor puede visualizar los cambios en la sección de consultar servicios.}
		\UCitem{Errores}{
		Que los campos no cumplan con los tipos de datos solicitados.
		No se efectua correctamente la operacion.}

		\UCitem{Tipo}{Caso de uso primario.}
		\UCitem{Observaciones}{}
		\UCitem{Autor}{Francisco Garcia Enriquez}
		\UCitem{Revisor}{Roberto mendoza Saavedra}
	\end{UseCase}

	\begin{UCtrayectoria}{Principal}
		\UCpaso Solicita ingreso al sistema.
		\UCpaso Toma la sesion del actor
		\UCpaso Mustra en la página principal el menú de opciones.
		\UCpaso[\UCactor] Selecciona del menú la opción servicios.
		\UCpaso Muestra las opciones que el actor puede realizar: Agregar servicio, consultar servicios, eliminar servicios, suspender servicios y actualizar servicios.
		\UCpaso[\UCactor] Selecciona la opción, Actualizar servicios. 
		\UCpaso carga una tabla con los servicios registrados, adjunto de un boton de edición.
		\UCpaso[\UCactor] Presiona el boton de editar, en el servicio que desea modificar.
		\UCpaso sistema muestra una página donde se cargan los datos del servicio en los campos del formulario. Plantilla UI Modificar Servicio.
		\UCpaso[\UCactor] modifica los datos que considere necesarios [Trayectoria B][Trayectoria C]
		\UCpaso[\UCactor] Confirma el registro y presiona el boton de actualizar datos.
		\UCpaso Actualiza los datos y muestra un mensaje MSG18- "los datos se han actualizado exitosamente."
		\UCpaso[\UCactor] puede regresar al menú de opciones mediante el boton de "regresar al menú de inicio".
	\end{UCtrayectoria}

		\begin{UCtrayectoriaA}{A}{No se cuenta con ningun registro.}
		\UCpaso muestra un mensaje MSG-7 "No existe ningun registro".
		\end{UCtrayectoriaA}

		\begin{UCtrayectoriaA}{B}{No se ingresaron los campos obligatorios.}
			\UCpaso[\UCactor] No llena algun campo del formulario y confirma la actualización.
			\UCpaso Muestra un mensaje MSG-3 "Ingresa los campos obligatorios marcados con * para continuar", campos obligatorios se muestran en Plantilla UI campos Obligatorios Servicios.
		\end{UCtrayectoriaA}

		\begin{UCtrayectoriaA}{C}{El dato no corresponde al tipo de dato esperado.}
			\UCpaso Muestra un mensaje MSG-4 "Ingresa el valor correcto de acuerdo al formato que aparece en el campo."
		\end{UCtrayectoriaA}
	
		%-------------------------------------- TERMINA descripción del caso de uso.