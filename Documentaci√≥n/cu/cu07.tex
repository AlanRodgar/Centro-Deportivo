% \IUref{IUAdmPS}{Administrar Planta de Selección}
% \IUref{IUModPS}{Modificar Planta de Selección}
% \IUref{IUEliPS}{Eliminar Planta de Selección}

%\begin{UseCase}[archivo de imágen]{UCX}{Nombre del Caso de uso}{
	\begin{UseCase}{CU7.0}{Registrar áreas}{En esta sección el gerente de sucursal podrá dar de alta una nueva área.}
		\UCitem{Versión}{1.0}
		\UCitem{Actor}{Gerente de sucursal}
		\UCitem{Propósito}{Agregar nuevas áreas al sistema para realizar alguna actividad.}
		\UCitem{Entradas}{Nombre del Área, tipo de área, largo, ancho, capacidad, responsable y descripción.}
		\UCitem{Origen}{Desde el teclado.}
		\UCitem{Salidas}{{\bf MS19-}``El registro fue exitoso.''}
		\UCitem{Destino}{Los datos serán almacenados en el sistema y se verán reflejados en la tabla de la sección de consultar áreas.}
		\UCitem{Precondiciones}{Que no esté esa área registrada previamente con el mismo nombre.Que exista alguna actividad que pueda utilizar dicha área.}
		\UCitem{Postcondiciones}{El área registrada se podrá visualizar en la sección de consultar áreas}
		\UCitem{Errores}{
{\bf E3:} ``No se ingresaron todos los campos obligatorios.'' -- El sistema muestra el Mensaje {\bf MSG3-}``Ingresa los campos obligatorios marcados con * para continuar.'' mostrados en \IUref{IU8}{Pantalla de campos obligatorios de áreas.} y continua con el paso 10.
{\bf E4:} ``El formato del dato es incorrecto''  -- El sistema muestra el Mensaje {\bf MSG4-}``Ingresa el valor correcto de acuerdo al formato que aparece en el campo.''
{\bf E8:} ``Nombre de área ya existente'' -- El sistema muestra el Mensaje {\bf MSG8-}``El nombre de área ya se encuentra registrado.'' }
		\UCitem{Tipo}{Caso de uso primario}
		\UCitem{Observaciones}{}
		\UCitem{Autor}{Francisco García Enríquez.}
		\UCitem{Revisor}{Martin Carrillo.}
	\end{UseCase}


	\begin{UCtrayectoria}{Principal}
		\UCpaso[\UCactor] Solicita el ingreso al apartado de clientes seleccionando la opción ``Clientes'' de la \IUref{IU6}{Pantalla de perfil de empleado}.
		\UCpaso Toma la sesión de gerente de sucursal.
		\UCpaso Muestra en la página principal del gerente de sucursal, las opciones que tiene disponibles para realizar operaciones el actor. \IUref{IU8}{Pantalla de menú inicial.} 
		\UCpaso[\UCactor] Selecciona del menú, la opción Áreas.
		\UCpaso Muestra las opciones que el gerente pueda realizar.
		\UCpaso[\UCactor] Selecciona la opción de Registrar Áreas. \IUref{IU18}{Pantalla del menu de áreas} 
		\UCpaso Mostrará \IUref{IU19}{Pantalla para registro de áreas}  para registrar un área.
		\UCpaso[\UCactor] Ingresa el nombre del área.
		\UCpaso Carga las áreas dispobinles desde la base de datos en la lista tipo de área.
		\UCpaso[\UCactor] Selecciona el tipo de área desde una lista desplegable.
		\UCpaso[\UCactor] Ingresa el dato en el campo ancho.
		\UCpaso[\UCactor] Ingresa el dato en el campo capacidad.
		\UCpaso[\UCactor] Ingresa el dato en el campo largo. 
		\UCpaso[\UCactor] Ingresa el nombre en el campo responsable. 
		\UCpaso[\UCactor] Confirma el registro, presionando el botón de registrar área.\Trayref{A} \Trayref{B} \Trayref{C}.
		\UCpaso Hace el registro y muestra una pantalla con un {\bf MS19-}``El registro fue exitoso.'' \IUref{IU8}{Pantalla de registro exitoso.}
	\end{UCtrayectoria}
		
		\begin{UCtrayectoriaA}{A}{El gerente de sucursal introduce espacios en blanco al inicio o caracteres especiales en el campo.}
			\UCpaso[\UCactor] introduce espacios en blanco al inicio o caracteres especiales en el campo nombre.
			\UCpaso mostrará {\bf MS4-}``Ingresa el valor correcto de acuerdo al formato que aparece en el campo, mostrados en la \IUref{IU19}{Pantalla para registro de áreas}.''
			\UCpaso[\UCactor] vuelve al campo correspondiente para corregir la información.
		\end{UCtrayectoriaA}

		\begin{UCtrayectoriaA}{B}{El gerente de sucursal introduce tipo de dato incorrecto.}
			\UCpaso[\UCactor] introduce espacios en blanco al inicio o caracteres especiales en el campo nombre.
			\UCpaso mostrará {\bf MS4-}``Ingresa el valor correcto de acuerdo al formato que aparece en el campo, mostrados en la \IUref{IU19}{Pantalla para registro de áreas}.''
			\UCpaso[\UCactor] vuelve al campo correspondiente para volver a ingresar el dato.
		\end{UCtrayectoriaA}

		\begin{UCtrayectoriaA}{C}{El gerente de sucursal ingresa un nombre ya existente.}
			\UCpaso[\UCactor] introduce un nombre ya existente.
			\UCpaso mostrará {\bf MS8-}``El nombre de área ya se encuentra registrado.''
		\end{UCtrayectoriaA}