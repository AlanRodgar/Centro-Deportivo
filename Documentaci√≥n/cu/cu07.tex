% \IUref{IUAdmPS}{Administrar Planta de Selección}
% \IUref{IUModPS}{Modificar Planta de Selección}
% \IUref{IUEliPS}{Eliminar Planta de Selección}

%\begin{UseCase}[archivo de imágen]{UCX}{Nombre del Caso de uso}{
	\begin{UseCase}{CU7.0}{Registrar áreas}{En esta sección el administrador podrá dar de alta una nueva área.}
		\UCitem{Versión}{1.0}
		\UCitem{Actor}{Gerente de sucursal}
		\UCitem{Propósito}{Que se agreguen nuevas áreas para realizar alguna actividad.}
		\UCitem{Entradas}{Nombre del Área, Tipo de área, largo, ancho, capacidad, responsable y descripción.}
		\UCitem{Origen}{Desde el teclado.}
		\UCitem{Salidas}{Mensaje de registro exitoso.}
		\UCitem{Destino}{Los datos serán enviados a la tabla con las áreas registradas en el área de consulta.}
		\UCitem{Precondiciones}{Que no esté esa área registrada previamente con el mismo nombre.Que exista alguna actividad que pueda utilizar dicha área.}
		\UCitem{Postcondiciones}{El área registrada se verá reflejada en la sección de consultas.}
		\UCitem{Errores}{Que la información registrada no corresponda a los tipos de datos esperados. Error al no llenar los campos obligatorios.}
		\UCitem{Tipo}{Caso de uso primario}
		\UCitem{Observaciones}{}
		\UCitem{Autor}{Francisco García Enríquez.}
		\UCitem{Revisor}{Martin Carrillo.}
	\end{UseCase}


	\begin{UCtrayectoria}{Principal}
		\UCpaso[\UCactor] Ingresa al sistema mediante el formulario de login.
		\UCpaso toma la sesión de gerente de sucursal.
		\UCpaso muestra en la página principal del gerente de sucursal, las opciones que tiene disponibles para realizar operaciones el actor. 
		\UCpaso[\UCactor] selecciona del menú, la opción Áreas.
		\UCpaso muestra las opciones que el gerente pueda realizar.
		\UCpaso[\UCactor] selecciona la opción de Registrar Áreas.
		\UCpaso mostrará la Plantilla UI 23 para registrar un área.
		\UCpaso[\UCactor] ingresa el nombre del área [TA].
		\UCpaso[\UCactor] confirma el registro presionando el botón de registrar.
		\UCpaso muestra el mensaje MSG: 45 “Campos obligatorios”  mostrados en la Plantilla UI 24 “Campos obligatorios”.
		\UCpaso[\UCactor] ingresa el dato correspondiente en el campo largo y confirma él envío [TB].
		\UCpaso manda un mensaje MSG: 45 “Campos obligatorios” .
		\UCpaso[\UCactor] ingresa el dato en el campo ancho [TB].
		\UCpaso[\UCactor] confirma el registro y presiona el botón registrar.
		\UCpaso muestra el mensaje MSG: 45 “Campos obligatorios”  mostrados en la Plantilla UI 24 “Campos obligatorios”.
		\UCpaso[\UCactor] confirma el registro presionando el botón registrar.
		\UCpaso manda un mensaje mensaje MSG: 45 “Campos obligatorios” , refiriéndose a  que el campo responsable debe ser ingresado.
		\UCpaso[\UCactor] ingresa en nombre en el campo responsable [TA].
		\UCpaso[\UCactor] confirma el registro, presionando el botón de registrar área.
		\UCpaso hace el registro y muestra una pantalla con un mensaje, diciendo que el registro fue exitoso.
	\end{UCtrayectoria}
		
		\begin{UCtrayectoriaA}{A}{El gerente de sucursal introduce espacios en blanco al inicio o caracteres especiales en el campo.}
			\UCpaso[\UCactor] introduce espacios en blanco al inicio o caracteres especiales en el campo nombre.
			\UCpaso mostrará MSG:12 
		\end{UCtrayectoriaA}