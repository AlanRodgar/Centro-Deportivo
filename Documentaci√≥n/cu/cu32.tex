% \IUref{IUAdmPS}{Administrar Planta de Selección}
% \IUref{IUModPS}{Modificar Planta de Selección}
% \IUref{IUEliPS}{Eliminar Planta de Selección}


% Copie este bloque por cada caso de uso:
%-------------------------------------- COMIENZA descripción del caso de uso.


%\begin{UseCase}[archivo de imágen]{UCX}{Nombre del Caso de uso}{
	\begin{UseCase}{CU32}{Consultar Servicios.}{
		Permite al gerente de sucursal, recepcionista, consultar los servicios con los que cnuenta la sucrusal.
	}
		\UCitem{Versión}{1.0}
		\UCitem{Actor}{Gerente de sucursal, Recepcionista}
		\UCitem{Propósito}{Consultar los servicios que se encuentran registrados en el sistema.}
		\UCitem{Entradas}{Desde el teclado o mouse.}
		\UCitem{Origen}{Los datos serán mostrados desde una tabla}
		\UCitem{Salidas}{Se mostrará el área del servicio con sus atributos, tales como: Nombre, ubicación, estado del servicio, costo, horario, responsable y descripción.}
		\UCitem{Destino}{Los datos serán mostrados desde una tabla.}
		\UCitem{Precondiciones}{Para visualizar un servicio se requiere que el servicio se encunetre registrado, tener un área registrada previamente y al menos un instructor registrado.}
		\UCitem{Postcondiciones}{El actor podrá regresar al menú de opciones para realizar otras operaciones dentro del sistema.}
		\UCitem{Errores}{Que el sistema no detecte datos incorrectos y se visualicen en la tabla de consulta.}
		\UCitem{Tipo}{Caso de uso primario.}
		\UCitem{Observaciones}{}
		\UCitem{Autor}{Francisco García Enríquez}
		\UCitem{Revisor}{Roberto Mendoza Saavedra}
	\end{UseCase}

	\begin{UCtrayectoria}{Principal}
		\UCpaso[\UCactor] El Solicita ingreso al sistema.
		\UCpaso Toma la sesion del actor.
		\UCpaso Mustra en la página principal el menú de opciones.
		\UCpaso[\UCactor] Mustra en la página principal el menú de opciones.
		\UCpaso Muestra las opciones que el actor puede realizar: Agregar servicio, Consultar servicios, Eliminar servicios, Suspender servicios y actualizar servicios.
		\UCpaso[\UCactor] Selecciona la opción, consultar servicios.
		\UCpaso Muestra los datos de los servicios registrados mediante una tabla.
		\UCpaso[\UCactor] puede visualizar los datos de los servicios registrados.
		\UCpaso[\UCactor] Finaliza su operacion dentro del área de conuslta.
		\UCpaso[\UCactor] Puede regresar al menu de opciones de la sección, mediante el boton de menú de inicio.
	\end{UCtrayectoria}

\begin{UCtrayectoriaA}{A}{No existe ningun dato registrado.}
			\UCpaso[\UCactor] Muestra el mensaje {\bf MS7-}``No existe ningun registro''
			\UCpaso[\UCactor] Finaliza su operación dentro del área de consulta. 
			\UCpaso[\UCactor] Puede regresar al menú de opciones de la sección \IUref{IU18}{Pantalla de menu de opciones áreas} mediante el botón de menú de inicio.
		\end{UCtrayectoriaA}
	
		%-------------------------------------- TERMINA descripción del caso de uso.