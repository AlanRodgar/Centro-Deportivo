% \IUref{IUAdmPS}{Administrar Planta de Selección}
% \IUref{IUModPS}{Modificar Planta de Selección}
% \IUref{IUEliPS}{Eliminar Planta de Selección}

% 


% Copie este bloque por cada caso de uso:
%-------------------------------------- COMIENZA descripción del caso de uso.

%\begin{UseCase}[archivo de imágen]{UCX}{Nombre del Caso de uso}{
	\begin{UseCase}{CU15}{Eliminar del sistema una sucursal.}{
		Elimina un registro de la base de datos.
	}
		\UCitem{Versión}{0.1}
		\UCitem{Actor}{Gerente de operación de negocio}
		\UCitem{Propósito}{Borrar de la base de datos un registro de una sucursal, debido a que hubo un error en el momento de dar de alta una sucursal.}
		\UCitem{Entradas}{Selección del registro a eliminar.}
		\UCitem{Origen}{El mouse del actor.}
		\UCitem{Salidas}{Se mostrará un mensaje de confirmación y posteriormente un mensaje de eliminación correcta o fallida.}
		\UCitem{Destino}{Los mensajes a mostrar se desplegaran en la pantalla del computador del actor. Los datos se enviarán al servidor para procesarlos.}
		\UCitem{Precondiciones}{El actor ingreso al sistema mediante un login. El sistema se encuentra en la pantalla que  despliega un listado de los registros agregados recientemente.}
		\UCitem{Postcondiciones}{En la base de datos del sistema se tendra un registro menos de una sucursal.}
		\UCitem{Errores}{1. El actor seleccionó una sucursal que no debía borrar}
		\UCitem{Tipo}{Caso de uso primario}
		\UCitem{Observaciones}{En ocaciones solo se introduce mal un solo dato. En lugar de borrar directamente el registro el sistema debe mostrar al actor si prefiere actualizar los datos referentes a la sucursal que seleccionó.}
		\UCitem{Autor}{Fernández Quiñones Isaac.}
		\UCitem{Revisó}{}
	\end{UseCase}

	\begin{UCtrayectoria}{Principal}
		\UCpaso[\UCactor] Ingresa a la plataforma web.
		\UCpaso Despliega la \IUref{IU23}{Pantalla de Control de Acceso} \label{CU15Login} para acceder al sistema.
		\UCpaso[\UCactor] Proporciona su userName y password. 
		\UCpaso Válida que el actor se encuentre dado de alta en el sistema. Se utiliza la regla \BRref{BR117}{Determinar si el usuario tiene acceso al sistema.} \Trayref{A}.
		\UCpaso Despliega la \IUref{IU99}{Pantalla dar de baja sucursal}. \label{CU15Sucursales}
		\UCpaso[\UCactor] Pone el mouse sobre la fila de la tabla correspondiente a la sucursal que quiere eliminar.
		\UCpaso[\UCactor] Presiona el botón izquierdo de su mouse.
		\UCpaso Muestra el mensaje {\bf MSG1-}``¿Está [{\em seguro}] de querer eliminar este registro.''. \Trayref{B}  \Trayref{C}
		\UCpaso Elimina el registro de la base de datos. \label{CU15Eliminar}
		\UCpaso Muestra el \IUref{UI88}{Mensaje de registro eliminado satisfactoriamente}.
	\end{UCtrayectoria}
		
		\begin{UCtrayectoriaA}{A}{El actor no cuenta con las credenciales válidas para poder ingresar al sistema.}
			\UCpaso Muestra el mensaje {\bf MSG1-}``Usuario [{\em y/o}] contraseñas no validos.''.
			\UCpaso[\UCactor] Oprimé el botón \IUbutton{Aceptar}.
			\UCpaso Continua en el paso \ref{CU11Login} del \UCref{CU15}.
		\end{UCtrayectoriaA}
		
		\begin{UCtrayectoriaA}{B}{El actor confirma la eliminación del registro}
			\UCpaso[\UCactor] Presionó el botón \IUbutton{Aceptar} del mensaje de confirmación.
			\UCpaso Continua en el paso \ref{CU15Eliminar} del \UCref{CU15}. 
		\end{UCtrayectoriaA}
		
		\begin{UCtrayectoriaA}{C}{El actor no desea la eliminación del registro}
			\UCpaso[\UCactor] Presionó el botón \IUbutton{Cancelar} del mensaje de confirmación.
			\UCpaso Continua en el paso \ref{CU15Sucursales} del \UCref{CU15}. 
		\end{UCtrayectoriaA}		
%-------------------------------------- TERMINA descripción del caso de uso.
%%%%%%%%%%%%%%%%%%%%%%%%%%%%%%%%%%%%%%