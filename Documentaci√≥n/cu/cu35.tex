% \IUref{IUAdmPS}{Administrar Planta de Selección}
% \IUref{IUModPS}{Modificar Planta de Selección}
% \IUref{IUEliPS}{Eliminar Planta de Selección}


% Copie este bloque por cada caso de uso:
%-------------------------------------- COMIENZA descripción del caso de uso.


%\begin{UseCase}[archivo de imágen]{UCX}{Nombre del Caso de uso}{
	\begin{UseCase}{CU35}{Registrar Servicios.}{
		Permite al gerente de sucursal, recepcionista, registrar un servicio en el sistema, añadiendo atributos como: nombre, tipo de servicio, areá en actividad, estado, responsables y descripción.
	}
		\UCitem{Versión}{1.0}
		\UCitem{Actor}{Gerente de Sucursal, Recepcionista}
		\UCitem{Propósito}{Dar de alta un servicio en el sistema, para poder inscribir dicho servicio en la adquisisción de membresias.}
		\UCitem{Entradas}{Nombre, ubicación, estado del servicio, costo, horario, responsable y descripción.}
		\UCitem{Origen}{Desde el teclado.}
		\UCitem{Salidas}{Se muestra un mensaje de "Servicio registrado".}
		\UCitem{Destino}{Los datos serán almacenados en el sistema y se verán reflejados en el área de consulta de servicios.}
		\UCitem{Precondiciones}{
		Que no exista un sevicio registrado previamente, con el mismo nombre.
		Que exista al menos una área registrada en el sistema para registrar dicho servicio.
		Que exista al menos un instructor que funga el papel de responsable.}
		\UCitem{Postcondiciones}{El servicio registrado se verá reflejado en la sección de consultar servicios.}
		\UCitem{Errores}{
		No se puede registrar un servicio sin que existan áreas e instructores disponibles.
		No se efectua correctamente la operacion.
		Que la información registrada no corresponda a los tipos de datos esperados.
		Que no se carguen en el sistema las áreas e instructores disponibles.}
		\UCitem{Tipo}{Caso de uso primario.}
		\UCitem{Observaciones}{}
		\UCitem{Autor}{Francisco Garcia Enriquez}
		\UCitem{Revisor}{Roberto mendoza Saavedra}
	\end{UseCase}

	\begin{UCtrayectoria}{Principal}
		\UCpaso Solicita ingreso al sistema.
		\UCpaso Toma la sesion del actor
		\UCpaso Mustra en la página principal el menú de opciones.
		\UCpaso[\UCactor] Selecciona del menú la opción servicios.
		\UCpaso Muestra las opciones que el actor puede realizar: Agregar servicio, consultar servicios, eliminar servicios, suspender servicios y actualizar servicios.
		\UCpaso[\UCactor] Selecciona la opción, Registrar servicios. 
		\UCpaso[\UCactor] Ingresa el tipo de servicio.
		\UCpaso Carga en la lista desplegable las áreas disponibles. \Trayref{C}
		\UCpaso[\UCactor] selecciona un área de la lista.
		\UCpaso[\UCactor] Define un estado en el campo estado del servicio.
		\UCpaso[\UCactor] Ingresa el costo del servicio.
		\UCpaso[\UCactor] Define el horario de disponibilidad del servicio.
		\UCpaso[\UCactor] Ingresa la descripción del servicio en el campo descripción.
		\UCpaso[\UCactor] Confirma el registro presionando el boton registrar servicio. \Trayref{A} \Trayref{B}
		\UCpaso registra el servicio y muestra un mensaje "Servicio registrado".
		\UCpaso[\UCactor] Puede regresar al menú de opciones mediante el boton de "regresar al menú de inicio"

	\end{UCtrayectoria}

		\begin{UCtrayectoriaA}{A}{Los datos no corresponden al tipo de dato esperado.}
		\UCpaso muestra el mensaje MS7- "Los datos no corresponden al formato especificado por en el formulario".
		\end{UCtrayectoriaA}

		\begin{UCtrayectoriaA}{B}{No se llenan los campos obligaorios.}
			\UCpaso muestra el mensaje MS7- "Falta llenar los campos obligatorios" mostrados en Plantilla de campos obligatorios.
		\end{UCtrayectoriaA}

		\begin{UCtrayectoriaA}{C}{No existen áreas registradas.}
			\UCpaso muestra el mensaje MS20- "No se puede registrar un servicio, sin que existan áreas registradas."
		\end{UCtrayectoriaA}
	
		%-------------------------------------- TERMINA descripción del caso de uso.