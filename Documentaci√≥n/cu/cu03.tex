% \IUref{IUAdmPS}{Administrar Planta de Selección}
% \IUref{IUModPS}{Modificar Planta de Selección}
% \IUref{IUEliPS}{Eliminar Planta de Selección}


% Copie este bloque por cada caso de uso:
%-------------------------------------- COMIENZA descripción del caso de uso.


%\begin{UseCase}[archivo de imágen]{UCX}{Nombre del Caso de uso}{
	\begin{UseCase}{CU3}{Actualizar datos del cliente.}{
		Ayuda a modificar los datos del cliente una vez que se requiera la actualización de los mismos.
	}
		\UCitem{Versión}{1.1}
		\UCitem{Actor}{Gerente de Operaciones, Gerente de Sucursal, Ejecutivo de Ventas, Recepcionista.}
		\UCitem{Propósito}{Poder modificar la información de un cliente cada vez que se requiera realizar una actualización en sus datos.}
		\UCitem{Entradas}{Nombre(s) del cliente, Apellido paterno, Apellido materno o CURP según la búsqueda del actor. Los datos que se van a actualizar, definidos por el actor.}
		\UCitem{Origen}{Teclado.}
		\UCitem{Salidas}{Mensaje de actualización exitosa, correo electrónico de aviso sobre cambios en su información.}
		\UCitem{Destino}{Pantalla y servidor de correo.}
		\UCitem{Precondiciones}{El cliente debe contar con un registro previo.}
		\UCitem{Postcondiciones}{Registro actualizado del cliente.}
		\UCitem{Errores}{{\bf E3:} ``No se ingresaron todos los campos obligatorios.'' -- El sistema muestra el Mensaje {\bf MSG3-}``Ingresa los campos obligatorios marcados con * para continuar'' y continua al paso 11.
		
		{\bf E4:} ``El formato del dato es incorrecto''. -- El sistema muestra el Mensaje {\bf MSG4-}``Ingresa el valor correcto de acuerdo al formato que aparece en el campo'' y continua en el paso 11.
		
		{\bf E5:} ``La clave CURP, nombre o apellidos no existen'' -- El sistema muestra el Mensaje {\bf MSG5-}``La clave CURP, nombre(s) o apellidos no existen. Ingresa un dato válido'' y continua al paso 5.}
		\UCitem{Tipo}{Caso de uso que extiende del \UCref{CU31}.}
		\UCitem{Observaciones}{Los datos a actualizar van en función al dato que quiera modificar el actor.}
		\UCitem{Autor}{Roberto Mendoza Saavedra}
		\UCitem{Revisor}{}
	\end{UseCase}

	\begin{UCtrayectoria}{Principal}
		\UCpaso[\UCactor] Solicita el ingreso al apartado de clientes, seleccionando la opción de ``Clientes'' que aparece en la \IUref{IU6}{Pantalla de perfil de empleado.}
		\UCpaso Muestra las operaciones disponibles para el actor mediante la \IUref{IU7}{Pantalla de operaciones del empleado.}
		\UCpaso[\UCactor] Solicita la actualización de datos del cliente seleccionando la opción de ``Actualizar/Baja'' de la \IUref{IU7}{Pantalla de operaciones del empleado.}
		\UCpaso Solicita el nombre(s), apellido paterno, materno o CURP del cliente mediante la \IUref{IU9}{Pantalla de Búsqueda de Cliente.}
		\UCpaso[\UCactor] Proporciona el nombre(s), apellido paterno, materno o CURP, dependiendo el filtro de búsqueda que desee.
		\UCpaso[\UCactor] Solicita la búsqueda del registro(s) presionando el botón \IUbutton{Buscar} de la \IUref{IU9}{Pantalla de Búsqueda de Cliente.}
		\UCpaso Verifica si los datos que se ingresaron coinciden con algún registro [E5].
		\UCpaso Muestra un registro breve del cliente(s) con los datos de Nombre, Apellido paterno, Apellido materno, CURP y las opciones de ``Editar'' y ``Eliminar'' a través de la \IUref{IU9}{Pantalla de Búsqueda de Cliente.}
		\UCpaso[\UCactor] Solicita la actualización de los datos del cliente seleccionando la opción de ``Editar'' de la \IUref{IU9}{Pantalla de Búsqueda de Cliente.}
		\UCpaso Muestra el formulario de la \IUref{IU3}{Pantalla de Registrar Cliente} para que el actor modifique los campos que considere necesarios. 
		\UCpaso[\UCactor] Proporciona los datos a actualizar. La modificación de los datos se realiza bajo la \BRref{BR100}{Control de la información del cliente.} 
		\UCpaso[\UCactor] Confirma la actualización de los datos presionando el botón \IUbutton{Guardar} de la \IUref{IU3}{Pantalla de Registrar Cliente.}
		\UCpaso Verifica que todos los campos marcados como obligatorios en la \IUref{IU3}{Pantalla de Registrar Cliente} se hayan ingresado [E3].
		\UCpaso Verifica que todos los datos proporcionados tengan el formato especificado en el modelo de entidades de acuerdo al tipo de dato de cada campo [E4].
		\UCpaso Actualiza la información del cliente y muestra el Mensaje {\bf MS18-}``Los datos se han actualizado exitosamente''.
		\UCpaso Direcciona a la \IUref{IU9}{Pantalla de Búsqueda de Cliente.}
	\end{UCtrayectoria}
	
		%-------------------------------------- TERMINA descripción del caso de uso.