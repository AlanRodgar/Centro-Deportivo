% \IUref{IUAdmPS}{Administrar Planta de Selección}
% \IUref{IUModPS}{Modificar Planta de Selección}
% \IUref{IUEliPS}{Eliminar Planta de Selección}


% Copie este bloque por cada caso de uso:
%-------------------------------------- COMIENZA descripción del caso de uso.


%\begin{UseCase}[archivo de imágen]{UCX}{Nombre del Caso de uso}{
	\begin{UseCase}{CU4}{Ingresar al sistema.}{
		Permite a un actor autenticarse mediante su nombre de usuario y contraseña para poder realizar operaciones dentro del sistema.
	}
		\UCitem{Versión}{1.0}
		\UCitem{Actor}{Gerente de Operaciones, Gerente de Sucursal, Recepcionista, Ejecutivo de Ventas, Instructor, Encargado de área, Cliente, Usario y RH.}
		\UCitem{Propósito}{Disponer de las funcionalidades del sistema en cualquier momento mediante el nombre de usuario y contraseña.}
		\UCitem{Entradas}{Nombre de usuario y contraseña.}
		\UCitem{Origen}{Teclado.}
		\UCitem{Salidas}{\IUref{IU6}{Pantalla de perfil de empleado.} o \IUref{IU5}{Perfil de usuario.}}
		\UCitem{Destino}{Pantalla.}
		\UCitem{Precondiciones}{El cliente debe contar con un registro previo.}
		\UCitem{Postcondiciones}{}
		\UCitem{Errores}{{\bf E16:} ``Nombre de usuario y/o contraseña son incorrectos'' -- El sistema muestra el Mensaje {\bf MSG16-}``El nombre de usuario y/o contraseña son incorrectos. Veríficalos.'' y continua al paso 3.}
		\UCitem{Tipo}{Caso de uso primario.}
		\UCitem{Observaciones}{}
		\UCitem{Autor}{Roberto Mendoza Saavedra}
		\UCitem{Revisor}{}
	\end{UseCase}

	\begin{UCtrayectoria}{Principal}
		\UCpaso[\UCactor] Solicita el ingreso al apartado de clientes, seleccionando la opción de ``Clientes'' que aparece en la \IUref{IU6}{Pantalla de perfil de empleado.}
		\UCpaso Muestra las operaciones disponibles para el actor mediante la \IUref{IU7}{Pantalla de operaciones del empleado.}
		\UCpaso[\UCactor] Solicita la baja del registro del cliente seleccionando la opción de ``Actualizar/Baja de la \IUref{IU7}{Pantalla de operaciones del empleado.}
		\UCpaso Solicita el nombre(s), apellido paterno, materno o CURP del cliente mediante la \IUref{IU9}{Pantalla de Búsqueda de Cliente.}
		\UCpaso[\UCactor] Proporciona el nombre(s), apellido paterno, materno o CURP, dependiendo el filtro de búsqueda que desee.
		\UCpaso[\UCactor] Solicita la búsqueda del registro(s) presionando el botón \IUbutton{Buscar} de la \IUref{IU9}{Pantalla de Búsqueda de Cliente.}
		\UCpaso Verifica si los datos que se ingresaron coinciden con algún registro [E5].
		\UCpaso Muestra un registro breve del cliente(s) con los datos de Nombre, Apellido paterno, Apellido materno, CURP y las opciones de ``Editar'' y ``Baja'' a través de la \IUref{IU9}{Pantalla de Búsqueda de Cliente.}
		\UCpaso[\UCactor] Solicita la baja del registro del cliente seleccionando la opción de ``Baja'' de la \IUref{IU9}{Pantalla de Búsqueda de Cliente.} La baja de un registro está dada por la \BRref{BR100}{Control de la información del cliente.} 
		\UCpaso Muestra un mensaje de confirmación {\bf MS19-}``¿Estás seguro de dar de baja el registro?''.
		\UCpaso[\UCactor] Selecciona el botón \IUbutton{Aceptar} a través de la \IUref{IU9.2}{Pantalla de Búsqueda de Cliente-baja.}
		\UCpaso Inhabilita el registro seleccionado.
		\UCpaso Direcciona a la \IUref{IU9}{Pantalla de Búsqueda de Cliente.}
	\end{UCtrayectoria}
	
		%-------------------------------------- TERMINA descripción del caso de uso.