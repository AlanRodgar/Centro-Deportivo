% \IUref{IUAdmPS}{Administrar Planta de Selección}
% \IUref{IUModPS}{Modificar Planta de Selección}
% \IUref{IUEliPS}{Eliminar Planta de Selección}

%-------------------------------------- TERMINA descripción del caso de uso.

%\begin{UseCase}[archivo de imágen]{UCX}{Nombre del Caso de uso}{
	\begin{UseCase}{CU8.0}{Eliminar áreas}{
		En esta sección el gerente de sucursal podrá eliminar un área que se encuentre registrada, ya sea a causa de posibles errores como duplicación de contenido o en el registro de los datos.
	}
		\UCitem{Versión}{1.0}
		\UCitem{Actor}{Gerente de sucursal}
		\UCitem{Propósito}{Eliminar los registros del área del sistema en caso de ser necesario.}
		\UCitem{Entradas}{Nombre del área por medio de lista desplegable.}
		\UCitem{Origen}{Los datos serán digitados desde el teclado o bien seleccionados desde una lista desplegable.}
		\UCitem{Salidas}
		{
		En caso de que se haya eliminado el área:
		Mensaje de que el área ha sido eliminada exitosamente.
		En caso de que no haya ninguna área registrada:
		Mensaje de que no se encuentra registrada ninguna área.}

		\UCitem{Destino}{El área ya no aparecerá en la lista desplegable.
		En la sección consultas el área ya no aparecerá mas.}
		\UCitem{Precondiciones}{Que el área se encuentre registrada previamente.}
		\UCitem{Postcondiciones}{El registro ya no se encontrará en la sección de consultas.}
		\UCitem{Errores}{Que el sistema no cargue el área para ser eliminada.
		Que el sistema no realice la operación correspondiente }
		\UCitem{Tipo}{Caso de uso primario}
		\UCitem{Observaciones}{}
		\UCitem{Autor}{Francisco García Enríquez.}
		\UCitem{Revisor}{Martin Carrillo.}
	\end{UseCase}

\begin{UCtrayectoria}{Principal}
		\UCpaso[\UCactor] Solicita el ingreso al apartado de clientes seleccionando la opción ``Clientes'' de la \IUref{IU23}{Pantalla de perfil de empleado}.
		\UCpaso Toma la sesión de gerente de sucursal.
		\UCpaso Muestra en la página principal del gerente de sucursal, las opciones que tiene disponibles para realizar operaciones el actor. En la sección \IUref{IU18}{Pantalla de menu de opciones áreas}.
		\UCpaso[\UCactor] Selecciona del menú principal la opción Áreas.
		\UCpaso Muestra las opciones que el gerente pueda realizar.
		\UCpaso[\UCactor] Selecciona la opción eliminar áreas.
		\UCpaso Carga en una lista las áreas registradas \Trayref{A}.
		\UCpaso[\UCactor] Selecciona de la lista delegable el área que desea eliminar.
		\UCpaso[\UCactor] Confirma la operación y presiona el botón eliminar.
		\UCpaso Elimina el dato y muestra un mensaje {\bf MS7-}``El dato fue eliminado correctamente''.
		\UCpaso Muestra una opción para regresar al menú de opciones.En la sección \IUref{IU18}{Pantalla de menu de opciones áreas}.. 
	\end{UCtrayectoria}

\begin{UCtrayectoriaA}{A}{No existe ningun dato registrado.}
			\UCpaso[\UCactor] Muestra el mensaje {\bf MS7-}``No existe ningun registro''
			\UCpaso[\UCactor] Finaliza su operación dentro del área de consulta. 
			\UCpaso[\UCactor] Puede regresar al menú de opciones de la sección \IUref{IU18}{Pantalla de menu de opciones áreas} mediante el botón de menú de inicio.
		\end{UCtrayectoriaA}