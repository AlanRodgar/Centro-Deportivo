% \IUref{IUAdmPS}{Administrar Planta de Selección}
% \IUref{IUModPS}{Modificar Planta de Selección}
% \IUref{IUEliPS}{Eliminar Planta de Selección}

%-------------------------------------- TERMINA descripción del caso de uso.

%\begin{UseCase}[archivo de imágen]{UCX}{Nombre del Caso de uso}{
	\begin{UseCase}{CU8.0}{Eliminar áreas}{
		En esta sección el gerente podrá dar de baja un área que se encuentre registrada, ya sea a 				causa de posibles errores como duplicación de contenido o error a la hora del registro.
	}
		\UCitem{Versión}{1.0}
		\UCitem{Actor}{Gerente de sucursal}
		\UCitem{Propósito}{Que se agreguen nuevas áreas para realizar alguna actividad.}
		\UCitem{Entradas}{Nombre del área por medio de lista desplegable.}
		\UCitem{Origen}{Los datos serán digitados desde el teclado o bien seleccionados desde una 				lista desplegable.}
		\UCitem{Salidas}
		{
		En caso de que se haya eliminado el área:
		Mensaje de que el área ha sido eliminada exitosamente.
		En caso de que no haya ninguna área registrada:
		Mensaje de que no se encuentra registrada ninguna área.}

		\UCitem{Destino}{El área ya no aparecerá en la lista desplegable.
		En la sección consultas el área ya no aparecerá mas.}
		\UCitem{Precondiciones}{Que el área se encuentre registrada previamente.}
		\UCitem{Postcondiciones}{El registro ya no se encontrará en la sección de consultas.}
		\UCitem{Errores}{Que el sistema no cargue el área para ser eliminada.
		Que el sistema no realice la operación correspondiente }
		\UCitem{Tipo}{Caso de uso primario}
		\UCitem{Observaciones}{}
		\UCitem{Autor}{Francisco García Enríquez.}
		\UCitem{Revisor}{Martin Carrillo.}
	\end{UseCase}

\begin{UCtrayectoria}{Principal}
		\UCpaso[\UCactor] ingresa al sistema mediante el formulario  Plantilla UI:Login.
		\UCpaso toma la sesión de gerente de sucursal.
		\UCpaso muestra en la página principal del gerente de sucursal, las opciones que tiene disponibles para realizar operaciones el actor. 
		\UCpaso[\UCactor] selecciona del menú principal la opción Áreas.
		\UCpaso muestra las opciones que el gerente pueda realizar.
		\UCpaso[\UCactor] selecciona la opción eliminar áreas.
		\UCpaso carga en una lista las áreas registradas [TA].
		\UCpaso[\UCactor] selecciona de la lista delegable el área que desea eliminar.
		\UCpaso[\UCactor] confirma la operación y presiona el botón eliminar.
		\UCpaso elimina el dato y muestra un mensaje confirmando que el dato fue eliminado correctamente.
		\UCpaso muestra una opción para regresar al menú de opciones. 
	\end{UCtrayectoria}