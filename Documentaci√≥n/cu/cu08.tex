% \IUref{IUAdmPS}{Administrar Planta de Selección}
% \IUref{IUModPS}{Modificar Planta de Selección}
% \IUref{IUEliPS}{Eliminar Planta de Selección}

%-------------------------------------- TERMINA descripción del caso de uso.

%\begin{UseCase}[archivo de imágen]{UCX}{Nombre del Caso de uso}{
	\begin{UseCase}{CU8.0}{Eliminar áreas}{
		En esta sección el gerente de sucursal podrá eliminar un área que se encuentre registrada, ya sea a causa de posibles errores como duplicación de contenido o errores en el registro de los datos.
	}
		\UCitem{Versión}{1.0}
		\UCitem{Actor}{Gerente de sucursal}
		\UCitem{Propósito}{Eliminar los registros del área del sistema en caso de ser necesario.}
		\UCitem{Entradas}{Nombre del área por medio de lista desplegable.}
		\UCitem{Origen}{Los datos serán digitados desde el teclado o bien seleccionados desde una lista desplegable.}
		\UCitem{Salidas}
		{
		{\bf MSG20-}``Registro eliminado correctamente.''		
		{\bf MSG7-}``No existe ningun registro.'' }

		\UCitem{Destino}{El área ya no aparecerá en la lista desplegable.
		En la sección consultas el área ya no aparecerá mas.}
		\UCitem{Precondiciones}{Que el área se encuentre registrada previamente.}
		\UCitem{Postcondiciones}{El registro ya no se encontrará en la sección de consultas.}
		\UCitem{Errores}{ {\bf E07:} ``No se tiene ningún registro'' -- El sistema muestra el Mensaje {\bf MSG7-}``No existe ningun registro.''
		{\bf E06:} ``Error de eliminacion de registro no existente'' -- El sistema muestra el Mensaje {\bf MSG7-}``El registro no existe.''}
		\UCitem{Tipo}{Caso de uso primario}
		\UCitem{Observaciones}{}
		\UCitem{Autor}{Francisco García Enríquez.}
		\UCitem{Revisor}{Martin Carrillo.}
	\end{UseCase}

\begin{UCtrayectoria}{Principal}
		\UCpaso[\UCactor] Solicita el ingreso al apartado de clientes seleccionando la opción ``Clientes'' de la \IUref{IU23}{Pantalla de perfil de empleado}.
		\UCpaso Toma la sesión de gerente de sucursal.
		\UCpaso Muestra en la página principal del gerente de sucursal, las opciones que tiene disponibles para realizar operaciones el actor. En la sección \IUref{IU18}{Pantalla de menu de opciones inicial}.
		\UCpaso[\UCactor] Selecciona del menú principal la opción Áreas.
		\UCpaso Muestra las opciones que el gerente pueda realizar. \IUref{IU8}{Pantalla de menú de opciones de áreas.}
		\UCpaso[\UCactor] Selecciona la opción eliminar áreas.
		\UCpaso Carga en una lista las áreas registradas \Trayref{A}.
		\UCpaso[\UCactor] Selecciona de la lista delegable el área que desea eliminar. \IUref{IU8}{Pantalla de eliminacion de áreas.}
		\UCpaso[\UCactor] Confirma la operación y presiona el botón eliminar.
		\UCpaso Elimina el dato y muestra un mensaje {\bf MS20-}``El dato fue eliminado correctamente''.
		\UCpaso Muestra una opción para regresar al menú de opciones.En la sección \IUref{IU18}{Pantalla de menu de opciones de regreso}. 
	\end{UCtrayectoria}

\begin{UCtrayectoriaA}{A}{No existe ningun dato registrado.}
			\UCpaso[\UCactor] Muestra el mensaje {\bf MS7-}``No existe ningun registro''
			\UCpaso[\UCactor] Finaliza su operación dentro del área de consulta. 
			\UCpaso[\UCactor] Puede regresar al menú de opciones de la sección \IUref{IU18}{Pantalla de menu de opciones de regreso} mediante el botón de menú de inicio.
		\end{UCtrayectoriaA}