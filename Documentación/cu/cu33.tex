% \IUref{IUAdmPS}{Administrar Planta de Selección}
% \IUref{IUModPS}{Modificar Planta de Selección}
% \IUref{IUEliPS}{Eliminar Planta de Selección}


% Copie este bloque por cada caso de uso:
%-------------------------------------- COMIENZA descripción del caso de uso.


%\begin{UseCase}[archivo de imágen]{UCX}{Nombre del Caso de uso}{
	\begin{UseCase}{CU33}{Eliminar Servicios.}{
		Permite al gerente de sucursal, recepcionista, eliminar un servicio a causa de equivocación al registrar algun dato.
	}
		\UCitem{Versión}{1.0}
		\UCitem{Actor}{Gerente de sucursal, Recepcionista}
		\UCitem{Propósito}{Eliminar un servicio registrado en el sistema.}
		\UCitem{Entradas}{Desde el teclado o mouse y mediante una lista desplegable.}
		\UCitem{Origen}{Los datos serán mostrados desde una lista deplegable.}
		\UCitem{Salidas}{Se muestra un mensaje de "eliminación exitosa".}
		\UCitem{Destino}{El dato se eliminará del sistema y los datos ya no serán visibles en la sección de consulta de servicios.}
		\UCitem{Precondiciones}{Para eliminar un servicio se requiere que esté registrado previamente.}
		\UCitem{Postcondiciones}{El actor puede registrar nuevamente el servicio eliminado.}
		\UCitem{Errores}{{\bf E7:} ``No se tiene ningún registro.'' -- El sistema muestra el Mensaje {\bf MSG1-}``No existe ningun registro.'' y el actor puede regresar al menu principal. }
		\UCitem{Tipo}{Caso de uso primario.}
		\UCitem{Observaciones}{}
		\UCitem{Autor}{Roberto Mendoza Saavedra}
		\UCitem{Revisor}{Francisco García Enriquez}
	\end{UseCase}

	\begin{UCtrayectoria}{Principal}
		\UCpaso[\UCactor] Solicita el ingreso al apartado de clientes seleccionando la opción ``Clientes'' de la \IUref{IU6}{Pantalla de perfil de empleado}.
		\UCpaso Toma la sesión de gerente de sucursal.
		\UCpaso Muestra el menu de opciónes disponoibles para el actor.
		\UCpaso[\UCactor] Selecciona del menú la opción Servicios.
		\UCpaso Muestra las opciones que el gerente pueda realizar: Registrar Servicios, Consultar Servicios, Eliminar Servicios, Suspender Servicios y Actualizar Servicios. En la sección \IUref{IU18}{Pantalla de menu de opciones servicios}.
		\UCpaso[\UCactor] Selecciona la opción de Eliminar servicios.
		\UCpaso Carga una lista desplegable con los servicios registrados. \Trayref{A}
		\UCpaso[\UCactor] Selecciona de la lista desplegable el servicio que desea eliminar.
		\UCpaso[\UCactor] Confirma la operación y presiona el boton eliminar.
		\UCpaso Elimina el registro del sistema y muestra un mensaje MS7.-"El registro fue eliminado correctamente".
		\UCpaso[\UCactor] Puede regresar al menu de opciones de la sección, mediante el boton de menú de inicio.
	\end{UCtrayectoria}

	\begin{UCtrayectoriaA}{A}{No existe ningun dato registrado.}
		\UCpaso muestra el mensaje MS7- "No existe ningun registro".
		\UCpaso[\UCactor] Finaliza su operación dentro del área de consulta.
		\UCpaso[\UCactor] puede regresar al menú de opciones de la sección, mediante el boton de "regresar al menu de inició".
		\end{UCtrayectoriaA}
	
		%-------------------------------------- TERMINA descripción del caso de uso.